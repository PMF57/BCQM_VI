\documentclass[11pt]{article}

\usepackage[a4paper,margin=25mm]{geometry}
\usepackage[T1]{fontenc}
\usepackage[british]{babel}
\usepackage{lmodern}
\usepackage{microtype}
\usepackage{amsmath}
\usepackage{amssymb}
\usepackage{enumitem}
\setlist{nosep}

\title{BCQM VI Lab Note: Code Bring-up Runs and Early Phase-Diagram Signals}
\author{Peter M.~Ferguson \\ \textit{Independent Researcher}}
\date{26 January 2026}

\begin{document}
\maketitle

\section*{Purpose}
This note records the first end-to-end runs of the \texttt{bcqm\_vi\_spacetime} scaffold executed today, together with the observed regimes in the provisional temporal and spatial order-parameter proxies. The purpose is bring-up validation and qualitative mapping, \emph{not} a BCQM VI result claim.

\section*{Model status (scaffold)}
The current code is a primitives-only event-graph growth model (events + directed edges). Selection dynamics and ``lockstep'' detection are placeholders intended to:
\begin{itemize}
\item produce nontrivial spatial connectivity signals,
\item expose failure modes early (self-loops, star-collapse),
\item support later shaping toward the BCQM VI coupled-transition experiment design.
\end{itemize}

\section*{Workflow and artefacts}
Runs were performed locally. Each run produced:
\begin{itemize}
\item \texttt{RUN\_CONFIG\_\{run\_id\}.json} (resolved configuration and metadata),
\item \texttt{RUN\_METRICS\_\{run\_id\}.json} (scalar summaries).
\end{itemize}
For the bring-up scans, snapshots were disabled (\texttt{nosnaps}) to avoid excessive output volume.

\section*{Order-parameter proxies (current implementation)}
\begin{itemize}
\item \textbf{Temporal proxy:} \(Q_{\mathrm{clock}}\) and \(L = Q_{\mathrm{clock}}/\sqrt{N}\), with \(Q_{\mathrm{clock}}=\langle\Delta k\rangle/\sigma_{\Delta k}\) computed from a simple tick detector based on a synchrony threshold (scaffold proxy).
\item \textbf{Spatial proxy:} \(S_{\mathrm{perc}}\) (largest weakly connected component fraction in an active set \(V_{\mathrm{active}}\) defined by a recency window of length \(W_{\mathrm{coh}}\)); and \(S_{\mathrm{junc}}^{\mathrm{w}}\) (weighted junction statistic with \(\beta_{\mathrm{junc}}=1.5\)).
\item \textbf{Guard-rails:} \texttt{hubshare} and \texttt{max\_indegree}, with anomaly flags for star-collapse and runaway hubbing.
\end{itemize}

\section*{Runs performed (today)}
\subsection*{1. Quickcheck and anti-collapse patch}
A first quickcheck (N=32; \(n\in\{0,0.5,1\}\); one seed) revealed an extreme hub/self-loop collapse at \(n=1\) (star-like graph dominated by self-loops). A patch was applied:
\begin{itemize}
\item disallow self-loops (\(u\to u\) edges),
\item replace ``prefer the max-indegree node'' with a recency-weighted, anti-hub choice among a small top-\(k\) set.
\end{itemize}
The patched quickcheck eliminated the collapse: \texttt{hubshare} stayed small and \texttt{max\_indegree} remained \(\mathcal{O}(N)\).

\subsection*{2. Bring-up Phase 0 (nosnaps), single seed}
A Phase-0 bring-up scan was run with N=64, seed 1, and
\[
n\in\{0.0,0.2,0.4,0.6,0.8,1.0\}.
\]
All runs were stable (no anomaly flags). A non-monotone spatial response was observed: high connectivity at \(n=0.2\), low connectivity at \(n=0.4\)--0.6, then re-percolation at \(n=0.8\)--1.0. The temporal proxy turned on only weakly at \(n=1.0\).

\subsection*{3. Bring-up Phase 0 (nosnaps), three seeds}
The scan was repeated for seeds 1--3 (N=64). The qualitative behaviour above was reproducible across seeds:
\begin{itemize}
\item \(n=0.2\) is consistently a \emph{high} \(S_{\mathrm{perc}}\) regime with \(L=0\).
\item \(n=0.4\) and \(n=0.6\) are consistently \emph{low} \(S_{\mathrm{perc}}\) regimes with \(L=0\).
\item \(n=0.8\) is a \emph{re-percolated} regime (moderate/high \(S_{\mathrm{perc}}\)) still with \(L=0\).
\item \(n=1.0\) is a \emph{high} \(S_{\mathrm{perc}}\) regime and the first point where the temporal proxy turns on (small but nonzero \(Q_{\mathrm{clock}},L\), and \(\ell_{\mathrm{lock}}>0\)).
\end{itemize}

\section*{Recorded values (N=64; seeds 1--3, approximate ranges)}
\subsection*{Spatial regimes}
\begin{itemize}
\item \(n=0.0\): \(S_{\mathrm{perc}}\approx 0.64\)--0.67; \(S_{\mathrm{junc}}^{\mathrm{w}}\approx 0.27\)--0.29; \(L=0\).
\item \(n=0.2\): \(S_{\mathrm{perc}}\approx 0.85\)--0.86; \(S_{\mathrm{junc}}^{\mathrm{w}}\approx 0.49\)--0.52; \(L=0\).
\item \(n=0.4\): \(S_{\mathrm{perc}}\approx 0.18\)--0.19; \(S_{\mathrm{junc}}^{\mathrm{w}}\approx 0.05\)--0.06; \(L=0\).
\item \(n=0.6\): \(S_{\mathrm{perc}}\approx 0.21\)--0.25; \(S_{\mathrm{junc}}^{\mathrm{w}}\approx 0.12\); \(L=0\).
\item \(n=0.8\): \(S_{\mathrm{perc}}\approx 0.62\)--0.77; \(S_{\mathrm{junc}}^{\mathrm{w}}\approx 1.56\)--2.38; \(L=0\).
\item \(n=1.0\): \(S_{\mathrm{perc}}\approx 0.81\)--0.83; \(S_{\mathrm{junc}}^{\mathrm{w}}\approx 18\)--19; \(L\approx 0.04\) (nonzero).
\end{itemize}

\subsection*{Guard-rails}
Across all runs reported above:
\begin{itemize}
\item hub concentration remained small (\texttt{hubshare} \(\sim 10^{-4}\) to \(10^{-3}\)),
\item \texttt{max\_indegree} remained \(\mathcal{O}(N)\),
\item anomaly flags for star-collapse and runaway hubbing remained \texttt{false}.
\end{itemize}

\section*{Interpretation}
\subsection*{Pipeline readiness}
The pipeline is now stable and usable: outputs are compact in \texttt{nosnaps} mode; collapse failure modes were detected and corrected; \(S_{\mathrm{perc}}\) shows clear regime sensitivity.

\subsection*{Emergent ``space'' signal}
Even at scaffold stage, \(S_{\mathrm{perc}}\) behaves like a percolation-like connectivity order parameter on an event graph: it takes distinct values in distinct regimes and is reproducible across seeds. This supports the view that we are on the correct track for a graph-based ``space/connectivity'' emergence signal.

\subsection*{Time order and \(L\)}
Causal order exists throughout by construction (directed reachability). The temporal proxy \(L\) is intended to detect a stronger notion: clock-like cadence/lockstep coherence. In the current scaffold, \(L\) is zero across most of the scan and turns on only weakly at \(n=1.0\). At Phase 0 this is informative rather than problematic: it indicates that either the tick/lockstep proxy is too strict or misaligned, and/or that the coupling between synchrony and cross-link formation is not yet shaped to enforce a single coupled transition.

\subsection*{Non-monotone spatial response}
The reproducible dip (high \(S\) at \(n=0.2\), low \(S\) at \(n=0.4\)--0.6, high again at \(n=0.8\)--1.0) shows that the current mapping \(n\mapsto\) behaviour is not yet a clean monotone control knob. This is expected at scaffold stage and provides a concrete target for refinement.

\section*{Proposed next diagnostic step (deferred)}
Before moving to larger sizes (N=128+) or Phase-1 coarse scans, enable optional binned time-series for a small diagnostic set (seed 1 only) at \(n\in\{0.2,0.6,1.0\}\) to distinguish steady-state regimes from transient/burn-in artefacts and to guide the minimal rule modification required for the intended coupled-transition experiment design.



\section*{Addendum: Binned time-series probe (nosnaps)}
Following the Phase-0 bring-up scan, an additional diagnostic run set was executed with binned time-series enabled (120 bins over the measurement window) at
\[
n\in\{0.2,\ 0.6,\ 1.0\},
\]
with N=64 and seed 1, and snapshots disabled. The intent was to distinguish steady regimes from intermittent/coarsening behaviour that can be obscured by end-of-run scalar summaries.

\subsection*{Results (time-series duty-cycles)}
Let \(\mathrm{DC}_{0.5}\) denote the fraction of bins with \(S_{\mathrm{perc}}(t) > 0.5\), and \(\mathrm{DC}_{0.8}\) the fraction with \(S_{\mathrm{perc}}(t) > 0.8\). Also record \(\overline{S}_{\mathrm{perc}}\) as the mean over bins.

\begin{itemize}
\item \(n=0.2\): \(\overline{S}_{\mathrm{perc}}\approx 0.335\), \(\mathrm{DC}_{0.5}\approx 25.4\%\), \(\mathrm{DC}_{0.8}\approx 10.7\%\); \(L(t)\equiv 0\) in all bins.
\item \(n=0.6\): \(\overline{S}_{\mathrm{perc}}\approx 0.35\), \(\mathrm{DC}_{0.5}\approx 19.7\%\), \(\mathrm{DC}_{0.8}\approx 9.8\%\); \(L(t)\equiv 0\) in all bins.
\item \(n=1.0\): \(\overline{S}_{\mathrm{perc}}\approx 0.804\), \(\mathrm{DC}_{0.5}\approx 98.4\%\), \(\mathrm{DC}_{0.8}\approx 72.1\%\); \(L(t)\) becomes nonzero and stabilises in late bins (last-10-bin mean consistent with the final scalar \(L\approx 0.04\)).
\end{itemize}

\subsection*{Interpretation}
These time-series confirm that:
\begin{itemize}
\item for \(n\le 0.6\), the spatial connectivity proxy \(S_{\mathrm{perc}}\) is intermittent and/or still coarsening within the measurement window, so the final scalar \(S_{\mathrm{perc}}\) may overstate ``typical'' connectivity in that regime;
\item at \(n=1.0\), the high-connectivity regime is robust (high duty-cycle of percolation) and this is also the first point where the current temporal proxy \(L\) becomes nonzero and remains stable in late bins.
\end{itemize}
Therefore, where binned time-series are enabled for diagnostic runs near candidate transition points, scan-level reporting should include time-averaged or duty-cycle versions of \(S_{\mathrm{perc}}\) (and a late-bin summary for \(L\)) rather than relying solely on end-of-run scalar values.



\section*{Addendum: 50-bin baseline time-series probe (nosnaps)}
A lightweight baseline probe was run with binned time-series enabled (\texttt{timeseries\_bins}=50) at
\[
n\in\{0.2,\ 0.4,\ 0.6,\ 0.8,\ 1.0\},
\]
with \(N=64\) and seed 1, and snapshots disabled. The purpose is a compact baseline diagnostic for future tuning iterations (duty-cycle and mean-over-bins metrics).

Define \(\overline{S}_{\mathrm{perc}}\) as the mean over bins, and duty-cycles \(\mathrm{DC}_{0.5}\) and \(\mathrm{DC}_{0.8}\) as the fraction of bins with \(S_{\mathrm{perc}}(t)>0.5\) and \(>0.8\) respectively. For the temporal proxy, record \(L_{\mathrm{late}}\) as the mean of the last 10 bins of \(L(t)\).

\begin{center}
\begin{tabular}{rcccc}
\hline
\(n\) & \(\overline{S}_{\mathrm{perc}}\) & \(\mathrm{DC}_{0.5}\) & \(\mathrm{DC}_{0.8}\) & \(L_{\mathrm{late}}\) \\
\hline
0.2 & 0.329 & 25.5\% & 11.8\% & 0 \\
0.4 & 0.332 & 23.5\% & 7.8\% & 0 \\
0.6 & 0.352 & 17.6\% & 7.8\% & 0 \\
0.8 & 0.433 & 15.7\% & 7.8\% & 0 \\
1.0 & 0.805 & 96.1\% & 68.6\% & 0.0393 \\
\hline
\end{tabular}
\end{center}

\subsection*{Interpretation}
This probe distinguishes ``high final \(S_{\mathrm{perc}}\)'' due to intermittent spikes from genuinely persistent percolation (high duty-cycle). It is therefore the preferred baseline diagnostic when iterating on the scaffold dynamics before committing to larger Phase-1 scans and ablations.



\section*{Addendum: BCQM V parity regression (C5 phase+cadence) via v\_glue engine}
The BCQM VI codebase was wired as a direct code descendant of BCQM V (\texttt{bcqm\_glue\_axes}) by porting \texttt{state.py}, \texttt{kernels.py}, and \texttt{metrics.py} as verbatim ancestor files with thin wrappers, and by implementing an \texttt{engine.mode=v\_glue} execution path that uses BCQM V kernels and the V metric \texttt{compute\_lockstep\_metrics}.
A C5 (phase+cadence) regression slice was then run at \(W_{\mathrm{coh}}\in\{20,50,100\}\) with ensembles (16--18 seeds) at \(N\in\{1,2,4,8\}\). Summaries below use robust statistics (mean, std, median, IQR, 10\% trimmed mean) as produced by \texttt{analysis/sweetspot\_check.py} (v0.2).
\subsection*{W\_coh = 20}
\begin{center}
\begin{tabular}{rccccccc}
\hline
\(N\) & count & mean & std & median & Q1 & Q3 & trim10 \\
\hline
1 & 16 & 0.0539 & 0.04 & 0.0434 & 0.0219 & 0.0836 & 0.0514 \\
2 & 16 & 0.305 & 0.062 & 0.314 & 0.251 & 0.34 & 0.306 \\
4 & 16 & 0.286 & 0.045 & 0.295 & 0.264 & 0.319 & 0.289 \\
8 & 16 & 0.289 & 0.051 & 0.284 & 0.253 & 0.307 & 0.286 \\
\hline
\end{tabular}
\end{center}
\noindent Sweet-spot heuristic (mean): \textbf{FAIL/INCONCLUSIVE}; (median): \textbf{PASS}.

\subsection*{W\_coh = 50}
\begin{center}
\begin{tabular}{rccccccc}
\hline
\(N\) & count & mean & std & median & Q1 & Q3 & trim10 \\
\hline
1 & 16 & 0.0405 & 0.033 & 0.034 & 0.0149 & 0.0621 & 0.0368 \\
2 & 16 & 0.553 & 0.12 & 0.575 & 0.478 & 0.614 & 0.548 \\
4 & 16 & 0.563 & 0.1 & 0.543 & 0.483 & 0.668 & 0.567 \\
8 & 16 & 0.594 & 0.14 & 0.601 & 0.523 & 0.68 & 0.584 \\
\hline
\end{tabular}
\end{center}
\noindent Sweet-spot heuristic (mean): \textbf{FAIL/INCONCLUSIVE}; (median): \textbf{FAIL/INCONCLUSIVE}.

\subsection*{W\_coh = 100}
\begin{center}
\begin{tabular}{rccccccc}
\hline
\(N\) & count & mean & std & median & Q1 & Q3 & trim10 \\
\hline
1 & 2 & 0.157 & 0.019 & 0.157 & 0.147 & 0.167 & 0.157 \\
2 & 2 & 0.752 & 0.13 & 0.752 & 0.685 & 0.818 & 0.752 \\
4 & 18 & 0.988 & 0.22 & 0.973 & 0.88 & 1.08 & 1.01 \\
8 & 18 & 1.12 & 0.8 & 0.773 & 0.585 & 1.5 & 1.04 \\
\hline
\end{tabular}
\end{center}
\noindent Sweet-spot heuristic (mean): \textbf{FAIL/INCONCLUSIVE}; (median): \textbf{PASS}.

\subsection*{Interpretation}
The C5 regression shows clear \(W_{\mathrm{coh}}\)-dependence and, at \(W_{\mathrm{coh}}=100\), heavy-tailed behaviour at larger \(N\) (mean and median disagree). This supports treating median (typical behaviour) and trimmed means as the preferred regression targets when assessing the BCQM V ``sweet spot'' narrative under finite ensembles.



\section*{Addendum: BCQM V parity regression (C9 phase+domains+cadence) via v\_glue engine}
A second parity regression was performed using the BCQM V C9 configuration (phase lock + cadence + domains) at \(W_{\mathrm{coh}}\in\{20,50,100\}\) with 16 seeds and \(N\in\{2,4,6,8\}\) (including an additional evaluation point \(N=6\)). Results below are as summarised by \texttt{analysis/sweetspot\_check.py} v0.3.

\subsection*{W\_coh = 20}
\begin{center}
\begin{tabular}{rccccccc}
\hline
\(N\) & count & mean & std & median & Q1 & Q3 & trim10 \\
\hline
2 & 16 & 0.309 & 0.053 & 0.308 & 0.289 & 0.318 & 0.310 \\
4 & 16 & 0.305 & 0.053 & 0.307 & 0.280 & 0.333 & 0.308 \\
6 & 16 & 0.297 & 0.059 & 0.303 & 0.252 & 0.341 & 0.301 \\
8 & 16 & 0.283 & 0.059 & 0.273 & 0.240 & 0.323 & 0.282 \\
\hline
\end{tabular}
\end{center}
Local-peak heuristic at \(N=4\) vs neighbours \(\{2,6\}\): \textbf{FAIL} (mean) and \textbf{FAIL} (median).

\subsection*{W\_coh = 50}
\begin{center}
\begin{tabular}{rccccccc}
\hline
\(N\) & count & mean & std & median & Q1 & Q3 & trim10 \\
\hline
2 & 16 & 0.567 & 0.11 & 0.573 & 0.501 & 0.633 & 0.565 \\
4 & 16 & 0.605 & 0.13 & 0.628 & 0.578 & 0.652 & 0.607 \\
6 & 16 & 0.552 & 0.20 & 0.579 & 0.410 & 0.661 & 0.537 \\
8 & 16 & 0.508 & 0.12 & 0.521 & 0.433 & 0.583 & 0.509 \\
\hline
\end{tabular}
\end{center}
Local-peak heuristic at \(N=4\) vs neighbours \(\{2,6\}\): \textbf{PASS} (mean) and \textbf{PASS} (median).

\subsection*{W\_coh = 100}
\begin{center}
\begin{tabular}{rccccccc}
\hline
\(N\) & count & mean & std & median & Q1 & Q3 & trim10 \\
\hline
2 & 16 & 0.932 & 0.20 & 0.860 & 0.806 & 1.05 & 0.921 \\
4 & 16 & 0.982 & 0.30 & 1.01 & 0.837 & 1.29 & 0.999 \\
6 & 16 & 0.862 & 0.51 & 0.795 & 0.508 & 1.10 & 0.818 \\
8 & 16 & 1.31 & 1.20 & 0.984 & 0.714 & 1.31 & 1.17 \\
\hline
\end{tabular}
\end{center}
Local-peak heuristic at \(N=4\) vs neighbours \(\{2,6\}\): \textbf{PASS} (mean) and \textbf{PASS} (median).

\subsection*{Interpretation}
Domains remain a secondary modulator: they do not replace cadence+phase as the clock engine, but they reshape the \(N\)-dependence. In this C9 run, a local moderate-\(N\) optimum (at \(N=4\) relative to neighbours) appears at \(W_{\mathrm{coh}}=50\) and \(100\), while at \(W_{\mathrm{coh}}=20\) the optimum shifts toward smaller \(N\).



\section*{Addendum: Path A cross-links on top of v\_glue (C5, W=100, N=4, p\_reuse sweep)}
Path A reintroduces spatial cross-links by maintaining an event graph and permitting reuse (co-selection) of events from an active set \(V_{\mathrm{active}}(t)\) while preserving the BCQM V glue engine unchanged as the time/clock mechanism. A batch sweep was executed at \(W_{\mathrm{coh}}=100\), \(N=4\) with 8 seeds for three fixed reuse probabilities \(p_{\mathrm{reuse}}\in\{0.20,0.50,0.80\}\). Spatial observables were computed post-hoc on \(V_{\mathrm{active}}\) (events created within the last \(W_{\mathrm{coh}}\) ticks, unioned across the run, plus all current frontier events).

\subsection*{Space and islands vs \(p_{\mathrm{reuse}}\)}
\begin{itemize}
\item \(p_{\mathrm{reuse}}=0.20\): \(S_{\mathrm{perc}}\approx 0.822\pm 0.072\), \(S_{\mathrm{junc}}^{\mathrm{w}}\approx 0.145\pm 0.012\), and \(F_{\max}=0.25\) in 8/8 seeds with bundle histogram \(\{1:4\}\) (four singletons).
\item \(p_{\mathrm{reuse}}=0.50\): \(S_{\mathrm{perc}}\approx 0.901\pm 0.035\), \(S_{\mathrm{junc}}^{\mathrm{w}}\approx 0.857\pm 0.082\), and \(F_{\max}=0.25\) in 8/8 seeds with bundle histogram \(\{1:4\}\).
\item \(p_{\mathrm{reuse}}=0.80\): \(S_{\mathrm{perc}}\approx 0.968\pm 0.027\), \(S_{\mathrm{junc}}^{\mathrm{w}}\approx 4.50\pm 0.57\), and \(F_{\max}\approx 0.875\pm 0.22\) with median \(F_{\max}=1\). Bundle histogram \(\{4:1\}\) occurs in 6/8 seeds (dominant island), with two mixed cases.
\end{itemize}

\subsection*{Clock survival}
Across the same batches (v\_glue clock metrics), \(Q_{\mathrm{clock}}\) remains viable and broadly stable: mean and median \(Q_{\mathrm{clock}}\) vary only modestly across \(p_{\mathrm{reuse}}\) in this slice. Thus, in this first integrated test, increased spatial cross-linking is compatible with maintaining a good clock.

\subsection*{Interpretation}
Spatial connectivity (\(S_{\mathrm{perc}}\)) increases smoothly with \(p_{\mathrm{reuse}}\), but island formation (\(F_{\max}\)) exhibits a sharper transition: the system remains a set of four ``islands'' at \(p_{\mathrm{reuse}}=0.20\) and \(0.50\), then nucleates a dominant coherent island at \(p_{\mathrm{reuse}}=0.80\) in most seeds.



\section*{Addendum: Path A coupled time--space--islands sweep (C5, W=100; N=4 and N=8)}
A matched set of 8-seed batches was executed at \(W_{\mathrm{coh}}=100\) for \(N=4\) and \(N=8\) with \(p_{\mathrm{reuse}}\in\{0.20,0.50,0.80\}\). Each batch records v\_glue clock metrics and post-hoc spatial observables. Island coherence is reported as \(F_{\max}(w_\star)\) at multiple thresholds \(w_\star\in\{0.10,0.20,0.30\}\) to avoid scale dependence of a single fixed overlap threshold.

\subsection*{N=4 summary (medians)}
\begin{center}
\begin{tabular}{rcccc}
\hline
\(p_{\mathrm{reuse}}\) & \(Q_{\mathrm{clock}}\) & \(S_{\mathrm{perc}}\) & \(S_{\mathrm{junc}}^{\mathrm{w}}\) & \(F_{\max}(0.30)\) \\
\hline
0.20 & 0.902 & 0.826 & 0.147 & 0.25 \\
0.50 & 0.967 & 0.901 & 0.853 & 0.25 \\
0.80 & 0.997 & 0.979 & 4.64 & 1.00 \\
\hline
\end{tabular}
\end{center}
At \(N=4\), connectivity rises smoothly with \(p_{\mathrm{reuse}}\) while \(F_{\max}\) remains at 0.25 through \(p_{\mathrm{reuse}}=0.50\) and then jumps sharply at \(p_{\mathrm{reuse}}=0.80\).

\subsection*{N=8 summary (medians; multi-threshold islands)}
\begin{center}
\begin{tabular}{rcccccc}
\hline
\(p_{\mathrm{reuse}}\) & \(Q_{\mathrm{clock}}\) & \(S_{\mathrm{perc}}\) & \(S_{\mathrm{junc}}^{\mathrm{w}}\) & \(F_{\max}(0.10)\) & \(F_{\max}(0.20)\) & \(F_{\max}(0.30)\) \\
\hline
0.20 & 1.01 & 0.879 & 0.145 & 0.125 & 0.125 & 0.125 \\
0.50 & 1.39 & 0.901 & 0.884 & 0.125 & 0.125 & 0.125 \\
0.80 & 0.938 & 0.962 & 4.71 & 1.00 & 0.562 & 0.125 \\
\hline
\end{tabular}
\end{center}
At \(N=8\), connectivity and junction density increase with \(p_{\mathrm{reuse}}\). However, island coherence depends strongly on \(w_\star\): at high reuse, the system is fully bundled at \(w_\star=0.10\), mixed/intermittent at \(w_\star=0.20\), and fully fragmented at \(w_\star=0.30\).

\subsection*{Interpretation}
These runs make the ``spacetime islands'' picture operational. Space/connectivity can increase smoothly with reuse, while island coherence exhibits sharper transitions and can be threshold-sensitive at larger \(N\). Clock coherence remains viable across the sweep; at \(N=8\) and high reuse the \(Q_{\mathrm{clock}}\) distribution becomes broad (median vs IQR), consistent with bursty/heavy-tailed lock episodes.



\section*{Addendum: Path A scan-from-n (C5, W=100): two-step emergence and N dependence}
A first ``true'' Path A scan was executed with \texttt{space.p\_reuse\_mode=from\_n}, i.e.\ \(p_{\mathrm{reuse}}:=n\), using \(n\in\{0,0.2,0.4,0.6,0.8\}\) and 8 seeds (56791--56798). Runs were performed for \(W_{\mathrm{coh}}=100\) at \(N=4\) and \(N=8\). Per n, the analysis reports medians and IQRs for \(Q_{\mathrm{clock}}\), \(S_{\mathrm{perc}}\), \(S_{\mathrm{junc}}^{\mathrm{w}}\), and \(F_{\max}(w_\star)\) for \(w_\star\in\{0.10,0.20,0.30\}\).

\subsection*{Summary (qualitative)}
Across both N values, the scan exhibits a clear two-step ordering:
\begin{enumerate}[label=\arabic*.]
\item \textbf{Space percolation first:} \(S_{\mathrm{perc}}\) rises sharply from the disconnected baseline at \(n=0\) to large values by \(n\approx 0.2\)--0.4, while island coherence remains fragmented.
\item \textbf{Island coherence later:} \(F_{\max}(w_\star)\) rises only at higher \(n\) (typically \(n\approx 0.6\)--0.8), and does so earlier at more lenient thresholds (e.g.\ \(w_\star=0.10\)).
\end{enumerate}
The N dependence is nontrivial: at \(N=4\) the islands become coherent even at strict threshold \(w_\star=0.30\) by \(n=0.8\), whereas at \(N=8\) the \(n=0.8\) islands remain threshold-sensitive (full at \(w_\star=0.10\), mixed at \(w_\star=0.20\), and fragmented at \(w_\star=0.30\)). Clock coherence remains viable across the scan; in these runs it does not collapse when space percolates.



\section*{Addendum: Path A scan-from-n robustness at W=50 (C5, N=4 and N=8)}
The scan-from-n protocol (\(p_{\mathrm{reuse}}:=n\), \(n\in\{0,0.2,0.4,0.6,0.8\}\); 8 seeds) was repeated at \(W_{\mathrm{coh}}=50\) for \(N=4\) and \(N=8\). The qualitative outcome matches the \(W_{\mathrm{coh}}=100\) case:

\begin{enumerate}[label=\arabic*.]
\item \textbf{Space percolation first:} \(S_{\mathrm{perc}}\) rises sharply from the disconnected baseline at \(n=0\) by \(n\approx 0.2\)--0.4 while island coherence remains at its baseline value.
\item \textbf{Islands later:} \(F_{\max}(w_\star)\) rises only at larger \(n\) (typically \(n\approx 0.6\)--0.8), and does so first at lenient thresholds (e.g.\ \(w_\star=0.10\)). At \(n=0.8\), \(N=4\) exhibits substantial coherence even at stricter threshold \(w_\star=0.30\) (median \(F_{\max}\approx 0.75\)), while \(N=8\) remains threshold-sensitive (full at \(w_\star=0.10\), mixed at \(w_\star=0.20\), fragmented at \(w_\star=0.30\)).
\end{enumerate}

Thus, the two-step emergence (connected space first, coherent spacetime islands later) and its N-dependence are robust across at least two coherence horizons \(W_{\mathrm{coh}}\in\{50,100\}\) in this C5 v\_glue + Path A configuration.



\section*{Addendum: Ablation suite W=100 (nospace vs space-on vs glue-off)}
To address the natural referee question ``can space percolate without a clock?'' an additional ablation condition was run at \(W_{\mathrm{coh}}=100\) for \(N=4\) and \(N=8\) using the same scan-from-n grid \(n\in\{0,0.2,0.4,0.6,0.8\}\) and 8 seeds. Three conditions were compared:
\begin{itemize}
\item \textbf{nospace:} space layer disabled (baseline);
\item \textbf{space-on:} Path A with \(p_{\mathrm{reuse}}:=n\);
\item \textbf{glue-off:} space layer enabled but glue coherence disabled (\texttt{ablation.glue\_decohere=true}), with phase-lock and cadence couplings set to zero and hop noise increased (q\_base override).
\end{itemize}

\subsection*{Outcome (qualitative)}
\begin{itemize}
\item In the \textbf{nospace} condition, \(S_{\mathrm{perc}}=S_{\mathrm{junc}}^{\mathrm{w}}=0\) for all n and \(Q_{\mathrm{clock}}\) is constant across n. This confirms that the scan parameter does not leak into the v\_glue clock engine when space is disabled.
\item In the \textbf{glue-off} condition, \(S_{\mathrm{perc}}\) and \(S_{\mathrm{junc}}^{\mathrm{w}}\) still rise strongly with n (space percolates), but \(Q_{\mathrm{clock}}\) collapses to \(\mathcal{O}(10^{-2})\). Thus space/connectivity can emerge without a good clock, while clock quality requires glue coherence.
\item In the \textbf{space-on} condition, the two-step emergence is reproduced (percolation first, islands later) with N-dependent threshold sensitivity.
\end{itemize}
This completes the minimal mechanism proof: space percolation is driven by the cross-link layer, not by the clock metric, while coherent clocks are a glue-engine phenomenon.



\section*{Addendum: 23 January 2026 --- W=50 replication, glue-off mechanism proof, and geometry diagnosis}
This addendum records three developments.

\subsection*{W=50 replication of the scan-from-n result}
The scan-from-n protocol (\(p_{\mathrm{reuse}}:=n\)) was repeated at \(W_{\mathrm{coh}}=50\) for \(N=4\) and \(N=8\), confirming that the two-step emergence and N-dependent threshold sensitivity observed at \(W_{\mathrm{coh}}=100\) persist at this second coherence horizon.

\subsection*{Glue-off mechanism proof}
An additional ablation condition (\texttt{ablation.glue\_decohere=true} with \texttt{q\_base\_override}) was run alongside nospace and space-on. The glue-off condition shows that space percolation (\(S_{\mathrm{perc}}\)) and junctioning (\(S_{\mathrm{junc}}^{\mathrm{w}}\)) can rise strongly with n even while \(Q_{\mathrm{clock}}\) collapses to \(\mathcal{O}(10^{-2})\). This closes the referee-style mechanism loop: the scan parameter influences space through the cross-link layer, not by leaking into the clock engine.

\subsection*{Geometry diagnosis: ds invalidity as a structural signal}
An audited ds scan showed \texttt{ds\_valid=0} broadly (low r$^2$), motivating a structural diagnosis rather than fit tuning. A plateau + ball-growth diagnostic indicates that the active graph slices are finite and can mix rapidly (plateau comparable to \(1/|C|\)), and that ball growth is locally sparse but globally shortcut-rich. This explains why a stable manifold-like spectral dimension is not yet an appropriate descriptor for these active slices under the current \(V_{\mathrm{active}}\) definition.



\section*{Addendum: Dynamic islands time series and paper figure mapping (26 January 2026)}
This addendum records the new time-series capability for Path A runs and locks the current figure set for the BCQM VI manuscript.

\subsection*{Time-series capability (binned)}
A binned time series can now be recorded during a run by setting:
\begin{verbatim}
output:
  write_timeseries: true
  timeseries_bins: 80   # (typical)
\end{verbatim}
The engine records a list of time-stamped records (interval chosen automatically from the measurement window) containing:
\begin{itemize}
\item \(S_{\mathrm{perc}}(t)\) and \(S_{\mathrm{junc}}^{\mathrm{w}}(t)\),
\item \(|V_{\mathrm{active}}|(t)\) and the largest-component size \(|C|(t)\),
\item \(F_{\max}(t)\) at \(w_\star\in\{0.10,0.20,0.30\}\) (reported as \texttt{F\_max\_by\_wstar}).
\end{itemize}
This makes the statement ``spacetime islands can come and go'' operational: in the pilot regime (W=100, N=8, n=0.8) the connectivity order parameter is high and comparatively stable, while the intermediate-threshold island coherence \(F_{\max}(w_\star=0.20)\) fluctuates substantially in time.

\subsection*{Pilot and ensemble results used for figures}
Two tiers of time-series evidence are now available:
\begin{enumerate}[label=\arabic*.]
\item \textbf{Single-seed pilot} (W=100, N=8, n=0.8, seed 56796): demonstrates the full time series and illustrates threshold sensitivity (w=0.10 saturated; w=0.30 fragmented; w=0.20 dynamic).
\item \textbf{Five-seed ensemble} (W=100, N=8; seeds 56791--56795): produces median\(\pm\)IQR bands for \(S_{\mathrm{perc}}(t)\) and \(F_{\max}(w_\star=0.20)(t)\) at n=0.4 and n=0.8, showing that ``space stays on while islands fluctuate'' is not a single-trajectory artefact and differs systematically between the two n values.
\end{enumerate}

\subsection*{Paper figure mapping (locked)}
The current manuscript figure set is locked as follows (file names reflect the PDFs in \texttt{figs/}):
\begin{itemize}
\item \textbf{Fig. 2} (pilot: full overlay) \\
\texttt{fig\_timeseries\_islands\_W100\_N8\_n0p8\_seed56796.pdf}
\item \textbf{Fig. 2a} (pilot: islands-only; expands a portion of Fig.\ 2) \\
\texttt{fig\_2a\_islands\_only\_W100\_N8\_n0p8\_seed56796.pdf}
\item \textbf{Fig. 2b} (pilot: space vs islands; \(S_{\mathrm{perc}}(t)\) and \(F_{\max}(w_\star=0.20)(t)\)) \\
\texttt{fig\_2b\_space\_vs\_islands\_W100\_N8\_n0p8\_seed56796.pdf}
\item \textbf{Fig. 3} (ensemble: n=0.4; median\(\pm\)IQR bands) \\
\texttt{fig\_3\_space\_vs\_islands\_ensemble\_n0p4\_W100\_N8.pdf}
\item \textbf{Fig. 3b} (ensemble: n=0.8; median\(\pm\)IQR bands) \\
\texttt{fig\_3b\_space\_vs\_islands\_ensemble\_n0p8\_W100\_N8.pdf}
\end{itemize}
Colour conventions are locked to the Fig.\ 2 base palette for consistency across all panels (notably \(S_{\mathrm{perc}}\) and \(F_{\max}(w_\star=0.20)\)).



\section*{Addendum: Ball-growth geometry diagnostic (26 January 2026)}
A ball-growth ensemble diagnostic was run to characterise the event-graph ``space'' without relying on return-probability spectral-dimension fits (which are structurally invalid on finite, rapidly mixing active slices). For each configuration the largest connected component \(C\subset V_{\mathrm{active}}\) was identified and the mean ball volume \(|B(r)|\) was estimated over sampled roots, reported as the median\(\pm\)IQR of the \emph{fraction covered} \(|B(r)|/|C|\) versus graph radius \(r\).

\subsection*{Figure mapping}
\begin{itemize}
\item \textbf{Fig. 4} (W=100, N=8): \texttt{fig\_4\_ball\_growth\_frac\_ensemble\_W100\_N8.pdf}
\item \textbf{Fig. 4a} (W=100, N=4): \texttt{fig\_4a\_ball\_growth\_frac\_ensemble\_W100\_N4.pdf}
\end{itemize}
Each panel compares n=0.4 vs n=0.8 (5 seeds; median\(\pm\)IQR bands).

\subsection*{Conclusion}
Ball growth shows a clear structural transition with reuse pressure: increasing n from 0.4 to 0.8 makes the active event graph substantially more shortcut-rich (smaller effective diameter), in both \(N=8\) and \(N=4\). In the n=0.4 regime the median fraction covered grows comparatively slowly with r (``yarn-like'' local channels with limited global reach), while in the n=0.8 regime the median fraction covered rises much more rapidly with r (locally sparse but globally tightened by shortcuts). This supports the earlier diagnosis that spectral-dimension-from-return-probability is not an appropriate descriptor on these finite active slices: the geometry is better characterised as \emph{channels plus shortcuts} whose shortcut density is an order parameter driven by reuse pressure.

\end{document}
