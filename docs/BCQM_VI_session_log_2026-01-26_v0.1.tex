\documentclass[11pt]{article}

\usepackage[a4paper,margin=25mm]{geometry}
\usepackage[T1]{fontenc}
\usepackage[utf8]{inputenc}
\usepackage[british]{babel}
\usepackage{lmodern}
\usepackage{microtype}
\usepackage{csquotes}
\usepackage{amsmath}
\usepackage{amssymb}
\usepackage{xurl}
\usepackage{hyperref}
\usepackage[nameinlink,noabbrev]{cleveref}
\usepackage{enumitem}
\setlist{nosep}

\input{bcqm_macros_v6}

\title{BCQM VI Session Log\\Monday 26 January 2026 (v0.1)}
\author{Peter M.~Ferguson \\ \textit{Independent Researcher}}
\date{26 January 2026}

\begin{document}
\emergencystretch=2em
\sloppy
\maketitle

\section*{Timestamp}
Session start reference: \textbf{Monday 26/01/26 11:22} (Europe/London).

\section*{Purpose}
Record the day’s work from the above timestamp, focusing on: (i) binned time-series logging for dynamic islands; (ii) paper-grade ensemble plots for dynamic islands (N=8 and N=4); (iii) ball-growth ensemble as the primary geometry diagnostic; and (iv) repository lock-down scaffolding and reproducibility plumbing.

\section*{1. Code changes (high-level)}
\subsection*{1.1 Time-series logging in the v\_glue engine}
A binned time-series capability was added to the production v\_glue engine. When enabled by:
\begin{verbatim}
output:
  write_timeseries: true
  timeseries_bins: 80
\end{verbatim}
the run logs a list of time-stamped records containing (per bin):
\begin{itemize}
\item \(S_{\mathrm{perc}}(t)\), \(S_{\mathrm{junc}}^{\mathrm{w}}(t)\),
\item \(|V_{\mathrm{active}}|(t)\), and the largest-component size \(|C|(t)\),
\item \(F_{\max}(t)\) at \(w_\star\in\{0.10,0.20,0.30\}\) (multi-threshold).
\end{itemize}
A production indentation fault in the first patch iteration was corrected (v0.2), after which the pilot run executed cleanly.

\subsection*{1.2 Robust plotting scripts}
The ensemble plotting scripts were patched for:
\begin{itemize}
\item pandas handling (Series-to-scalar extraction bug fixed),
\item consistent colour mapping locked to the Fig.\ 2 base palette (\(S_{\mathrm{perc}}\) and \(F_{\max}(w_\star=0.20)\)),
\item stable figure numbering and titles (Fig.\ 3 / Fig.\ 3b; Fig.\ 3c / Fig.\ 3d).
\end{itemize}
The \texttt{pyproject.toml} dependencies were updated to include \texttt{pandas} and \texttt{matplotlib}.

\subsection*{1.3 Ball-growth ensemble pipeline}
A ``ball-growth only'' ensemble pipeline was used (no reliance on return-probability spectral-dimension fits). It reports median\(\pm\)IQR of the fraction covered \(|B(r)|/|C|\) versus radius \(r\) on the largest component, comparing n=0.4 vs n=0.8 for \(N=8\) and \(N=4\).

\subsection*{1.4 Repository scaffolding}
A repo scaffold was prepared in two forms:
\begin{itemize}
\item v0.1: included a \texttt{paper/} folder (later rejected to match prior BCQM repo style),
\item v0.2: root-level \texttt{FIGURE\_MAP.md}, \texttt{REPRODUCIBILITY.md}, \texttt{ZENODO.md}, leaving paper sources and figures local/cloud and the paper PDF on Zenodo.
\end{itemize}
The \texttt{.gitignore} ignores bulk outputs by default.

\section*{2. Runs executed and outputs produced}
\subsection*{2.1 Dynamic islands: pilot (single seed)}
A pilot time-series run was performed (W=100, N=8, n=0.8, seed 56796) with binned logging enabled. The pilot confirms:
\begin{itemize}
\item connectivity metrics remain high and comparatively stable over time,
\item intermediate-threshold coherence \(F_{\max}(w_\star=0.20)(t)\) fluctuates, while \(w_\star=0.10\) saturates and \(w_\star=0.30\) remains fragmented.
\end{itemize}
Paper figures locked from this pilot: Fig.\ 2, Fig.\ 2a, Fig.\ 2b (stored locally; file mapping recorded in \texttt{FIGURE\_MAP.md}).

\subsection*{2.2 Dynamic islands: ensemble (N=8)}
An ensemble of 5 seeds (56791--56795) was run at W=100, N=8, comparing n=0.4 and n=0.8. The ensemble scripts produced median\(\pm\)IQR band figures:
\begin{itemize}
\item Fig.\ 3 (N=8, n=0.4): \(S_{\mathrm{perc}}(t)\) vs \(F_{\max}(w_\star=0.20)(t)\),
\item Fig.\ 3b (N=8, n=0.8): idem.
\end{itemize}

\subsection*{2.3 Dynamic islands: ensemble (N=4)}
The same ensemble protocol was repeated for W=100, N=4 (5 seeds; n=0.4 and n=0.8), producing:
\begin{itemize}
\item Fig.\ 3c (N=4, n=0.4),
\item Fig.\ 3d (N=4, n=0.8).
\end{itemize}

\subsection*{2.4 Companion plots}
Two companion ensemble plots of the largest-component size \(|C|(t)\) were generated:
\begin{itemize}
\item Fig.\ 3e (N=8) and Fig.\ 3f (N=4), comparing n=0.4 vs n=0.8.
\end{itemize}

\subsection*{2.5 Ball-growth geometry: ensemble}
The ball-growth ensemble was run and summarised for W=100:
\begin{itemize}
\item Fig.\ 4 (N=8): \(|B(r)|/|C|\) vs r (median\(\pm\)IQR) comparing n=0.4 vs n=0.8,
\item Fig.\ 4a (N=4): idem.
\end{itemize}
Conclusion: increasing n from 0.4 to 0.8 produces a marked tightening (shortcut-rich connectivity) on the active slice for both N values.

\section*{3. Data hygiene and archiving}
The bulk intermediate artefacts (e.g.\ SNAPSHOT CSV/JSON) were deleted. After clean-up, the working set size was reduced to approximately 43\,MB while maintaining reproducibility (selftest passes). A selftest was run and confirmed to pass after clean-up. Larger rollback archives exist separately.

\section*{4. Paper drafting status}
A draft BCQM VI paper file existed but a later revision (v0.1.3) was rejected as too sparse/rough and deferred. Next step is to resume paper drafting from the preferred baseline version, incorporating the locked figure set and the Stage--1 narrative with appropriate density.

\section*{5. Next steps (deferred to next session)}
\begin{itemize}
\item Resume BCQM VI paper drafting with the locked figure set and the Stage--1 results narrative.
\item Consider Stage--2 geometry ``cloth'' definition beyond \(V_{\mathrm{active}}(t)\), using ball growth as the primary diagnostic.
\end{itemize}

\end{document}
